\documentclass{article}
\usepackage[utf8]{inputenc}
\usepackage{exercise}

\author{Francisco Treviño}
\title{Optimization in Economic Theory:\\Lagrange's Method}

\begin{document}
\maketitle

\begin{abstract}
Exercises to practice Lagrange's optimization method.
\end{abstract}

\paragraph{Excersise 2.1: The Cobbs-Douglas Utility Function}
\paragraph{}
Consider the consumer's problem with utility function $U$ defined by
  \begin{equation}
    U(x, y) = x^\alpha y^\beta
  \end{equation}
Show that it yields the constant-budget-shared demand functions
  \begin{equation}\label{df}
    x = \frac{\alpha I}{(\alpha + \beta)p}, \qquad y = \frac{\beta I}{(\alpha + \beta)q}
  \end{equation}
The budget constraint is
  \begin{equation}
    px + qy = I
  \end{equation}

\paragraph{}
We begin by constructing the Lagrangian $L$ using the utility function and the constraint

  \begin{equation}
     L(x, y, \lambda) = x^\alpha y^\beta + \lambda (I - px - qy)
  \end{equation}

And now proceed with the partial derivatives, that involve implicit derivation, and optimize the results:
  \begin{equation}
    \frac{\partial L}{\partial x} = \alpha x^{\alpha - 1} y^\beta - \lambda p = 0
  \end{equation}

  \begin{equation}
    \frac{\partial L}{\partial y} = \beta x^\alpha y^{\beta - 1} - \lambda q = 0
  \end{equation}

  \begin{equation}\label{lag}
    \frac{\partial L}{\partial \lambda} = I - px - qy = 0
  \end{equation}

If we clear for $\lambda$ in the two first partial derivatives we have:

  \begin{equation}
    \lambda = \frac{\alpha x^{\alpha - 1} y^\beta}{p}
  \end{equation}

  \begin{equation}
    \lambda = \frac{\beta x^\alpha y^{\beta - 1}}{q}
  \end{equation}

So we can find a relationship between $x$ and $y$ doing some Algebra, as shown:

  \begin{equation}
    \frac{\alpha x^{\alpha - 1} y^\beta}{p} = \frac{\beta x^\alpha y^{\beta - 1}}{q}
  \end{equation}

Therefore, multiplying both sides by $x^{1 - \alpha}$
  \begin{equation}
    \frac{\alpha y^\beta}{p} = \frac{\beta x  y^{\beta - 1}}{q}
  \end{equation}

and now, multiplying both sides by $y^{1 - \beta}$
  \begin{equation}
    \frac{\alpha y}{p} = \frac{\beta x}{q}
  \end{equation}

From there we find $x$ and $y$ relationships:
  \begin{equation}
   x = \frac{\alpha q}{\beta p}y, \qquad y = \frac{\beta p}{\alpha q}x
  \end{equation}

Now lets substitute these, one at the time, in the Lagrange multiplier partial derivative $\ref{lag}$, as follows:
  \begin{equation}
    I = px + qy
  \end{equation}

Lets begin with $x$:
  \begin{equation}
    I = px + q\frac{\beta p}{\alpha q}x
  \end{equation}

  \begin{equation}
    I = px + \frac{\beta p}{\alpha}x
  \end{equation}

  \begin{equation}
    I = px( 1 + \frac{\beta}{\alpha})
  \end{equation}

  \begin{equation}
    I = px \left( \frac{\alpha + \beta}{\alpha} \right)
  \end{equation}

So we lastly solve for $x$:
  \begin{equation}\label{x}
    x = \frac{\alpha I}{(\alpha + \beta)p}
  \end{equation}

Now we follow the same process for $y$:
  \begin{equation}
    I = p\frac{\alpha q}{\beta p}y + qy
  \end{equation}

  \begin{equation}
    I = qy \left( \frac{\alpha}{\beta} + 1 \right)
  \end{equation}

  \begin{equation}
    I = qy \left( \frac{\alpha + \beta}{\beta}\right)
  \end{equation}

Solving for $y$, we have:
  \begin{equation}\label{y}
    y = \frac{\beta I}{(\alpha + \beta)q}
  \end{equation}

\paragraph{}
So we proved that the Utility function yields the constant-budget-shared demand functions $\ref{df}$

\paragraph{}
Lastly, we find the lagrange multiplier, lets use a previously obtained result, from the Lagrangian partial derivatives:
  \begin{equation}
    \frac{\partial L}{\partial x} = \alpha x^{\alpha - 1} y^\beta - \lambda p = 0
  \end{equation}

There we substitute the results obtained for $x$ and $y$, $\ref{x}$ and $\ref{y}$ respectively, and so we have:
  \begin{equation}
    \alpha \left( \frac{\alpha I}{(\alpha + \beta)p} \right)^{\alpha - 1} \left( \frac{\beta I}{(\alpha + \beta)q} \right)^\beta - \lambda p = 0
  \end{equation}

  \begin{equation}
    \lambda = \frac{\alpha}{p} \left( \frac{\alpha I}{(\alpha + \beta)p} \right)^{\alpha - 1} \left( \frac{\beta I}{(\alpha + \beta)q} \right)^\beta
  \end{equation}

And finally:
 \begin{equation}
    \lambda = \frac{\alpha^\alpha \beta^\beta I^{\alpha + \beta - 1}}{(\alpha + \beta)^{\alpha + \beta - 1}p^\alpha q^\beta}
  \end{equation}

.\newline

\paragraph{Exercise 2.2: The Linear Expenditure System}

\paragraph{}

Let the utility function be defined as:
\begin{equation}
  U(x, y) = \alpha \ln (x - x_0) + \beta \ln (y - y_0)
\end{equation}

where $x_0$, $y_0$ are given constants, and
\begin{equation}\label{res}
  \alpha + \beta = 1
\end{equation}

The budget constraint is
  \begin{equation}
    px + qy = I
  \end{equation}

Show that the optimal expenditures on the two goods are linear functions of income and prices.

The utility function brings with it a rich range of possible optimum choice. The budget shares of the two goods can now vary systematically with income and prices. One good can be a necessity and the other a luxury (but neither good can be inferior since $\alpha$ and $\beta$ must be positive). But the expenditures still have a simple functional form. For these reasons, this specification was popular in the early empirical work on consumer demand.

\paragraph{}
Lets build the Langrangian:
\begin{equation}
  L(x, y, z) = \alpha \ln (x - x_0) + \beta \ln (y - y_0) + \lambda (I - px - py)
\end{equation}

And now calculate the partial derivatives, the optimal points that is:
\begin{equation}
  \frac{\partial L}{\partial x} = \frac{\alpha}{x - x_0} - \lambda p = 0
\end{equation}

So we solve for x:
\begin{equation}
  \alpha = \lambda p (x - x_0)
\end{equation}

therefore:
\begin{equation}
  x = \frac{\alpha}{\lambda p} + x_0
\end{equation}

Optimum for $y$:
\begin{equation}
  \frac{\partial L}{\partial y} = \frac{\beta}{y - y_0} - \lambda q = 0
\end{equation}

We now solve for y:
\begin{equation}
  \beta = \lambda q (y - y_0)
\end{equation}

therefore:
\begin{equation}
  y = \frac{\beta}{\lambda q} + y_0
\end{equation}

Optimum for $\lambda$:
\begin{equation}
  \frac{\partial L}{\partial \lambda} = I - px - qy = 0
\end{equation}

Lets substitute the found expressions of $x$ and $y$ in the last equation:
\begin{equation}
  I - p \left( \frac{\alpha}{\lambda p} + x_0 \right) - q \left( \frac{\beta}{\lambda q} + y_0 \right) = 0
\end{equation}

and do simplify:
\begin{equation}
  I - \frac{\alpha}{\lambda} - px_0 - \frac{\beta}{\lambda} - qy_0 = 0
\end{equation}

\begin{equation}
  \frac{1}{\lambda} (\alpha + \beta) = I - px_0 - qy_0
\end{equation}

Hence, due to restriction given above:
\begin{equation}\label{lam}
  \lambda = 1/(I - px_0 - qy_0)
\end{equation}

We now proceed to find two expressions for $\lambda$ in terms of $x$ and $y$ and find the value of them in terms of the parameters.

From the partial derivative of the Lagrangian with respect to $x$ we can easily see that
\begin{equation}
  \lambda = \frac{\alpha}{p(x - x_0)}
\end{equation}

we can equal this equation with \ref{lam}:
\begin{equation}
  \frac{\alpha}{p(x - x_0)} = \frac{1}{(I - px_0 - qy_0)}
\end{equation}


\begin{equation}
  px = \alpha I - \alpha px_0 - \alpha q y_0 + px_0
\end{equation}

\begin{equation}
  px = \alpha I - \alpha q y_0 - px_0(\alpha - 1)
\end{equation}

Remember the restriction \ref{res}, so we have:
\begin{equation}
  px = \alpha I - \alpha q y_0 + \beta px_0
\end{equation}

From the partial derivative of the Lagrangian with respect to $y$ we follow the same procedure as we did with $x$:

\begin{equation}
  \lambda = \frac{\beta}{q(y - y_0)}
\end{equation}

we can equal this equation with \ref{lam}:
\begin{equation}
   \frac{\beta}{q(y - y_0)} = \frac{1}{(I - px_0 - qy_0)}
\end{equation}

\begin{equation}
  qy = \beta I - \beta px_0 - \beta q y_0 + qy_0
\end{equation}

\begin{equation}
  qy = \beta I - \beta px_0 - qy_0(\beta - 1)
\end{equation}

Remember the restriction \ref{res}, so we have:

\begin{equation}
  qy = \beta I - \beta px_0 + \alpha qy_0
\end{equation}

So we now have both Optimal expenditures for $x$ and $y$

\end{document}
\begin{equation}
 test
\end{equation}







